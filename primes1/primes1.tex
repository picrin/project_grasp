\documentclass{article}
\include{preamble}
\title{Project GRASP -- primes 1}
\author{Adam Kurkiewicz\\
  \small{Univeristy of Glasgow}\\
  \small{adam@kurkiewicz.pl}}
\usepackage[utf8]{inputenc}
\begin{document}
  \maketitle
  \section{Administrative Stuff}
  
  \subsection{Mailing list}
  Please sign up to the mailing list: https://groups.google.com/forum/\#!forum/project-grasp.
  \subsection{British Mathematical Olympiad}

  \section{Problems}
  
  \begin{enumerate}
    \item Bezout's lemma. Let $a$ and $b$ be any natural numbers, with $\gcd(a, b) = 1$. Show that there exist integers $n$ and $k$, such that $a\cdot n + b\cdot k = 1$.
    \item Prove Euclid's lemma. A prime number $p$ divides $ab$ if and only if $p$ divides $a$ or $p$ divides $b$.
    \item State and prove Fundamental Theorem of Arithmetic. The idea behind the theorem is that most natural numbers can be decomposed into prime factors, which are in some sense unique. But it's a good exercise to try and "dress up" it into a good wording. You should read a treatment of this subject on Timothy Gower's (a prominent British Mathematician) blog: https://gowers.wordpress.com/2011/11/18/proving-the-fundamental-theorem-of-arithmetic/
    \item Let $a$ and $b$ be any positive integers, and let $p$ be any prime number $p > 2$, such that $p$ divides $a + b$ and p divides $a^{2} + b^{2}$. Show that $p^{2}$ divides $a^{2} + b^{2}$.
    \item A positive integer is called charming if it is equal to 2 or is of the form $3^{i}×5^{j}$, where $i$ and $j$ are non-negative integers. Prove that every positive integer can be written as a sum of different charming integers.
    \item Let $m$ and $n$ be such integers, that in the set $\{1, 2, \ldots, n\}$ there are exactly $m$ prime numbers. Show that, among any $m + 1$ numbers from this set one can find a number which is a divisor of the product of the other $m$ numbers.
  \end{enumerate}
\end{document}
